\documentclass[a4paper,11pt,titlepage]{jsarticle}
\usepackage[dvipdfmx]{graphicx}
\usepackage[top=25mm, bottom=25mm, left=25mm, right=25mm]{geometry}
\usepackage{listings, jlisting}
\usepackage{amsfonts,amsmath,amsthm,mathtools,amssymb,physics}
\usepackage{comment}
\usepackage{url}

\mathtoolsset{showonlyrefs=true}
\numberwithin{equation}{section}

\newcommand{\midrad}[1]{\left\langle #1 \right\rangle}

\theoremstyle{definition}
\newtheorem{dfn}{定義}[section]
\newtheorem{thm}{定理}[section]

\renewcommand{\proofname}{\textbf{証明}}

\renewcommand{\lstlistingname}{ソースコード}
\lstset{%
    language={C},
    basicstyle={\small},%
    identifierstyle={\small},%
    commentstyle={\small\itshape},%
    keywordstyle={\small\bfseries},%
    ndkeywordstyle={\small},%
    stringstyle={\small\ttfamily},
    frame={tb},
    breaklines=true,
    columns=[l]{fullflexible},%
    xrightmargin=0zw,%
    xleftmargin=3zw,%
    numberstyle={\scriptsize},%
    numbersep=1zw,%
    lineskip=-0.5ex%
}

\begin{document}

\title{\LaTeX マニュアル}
\author{関根研究室}
\date{\today}
\maketitle

\tableofcontents

\newpage

\section{この文書について}

このマニュアルは \LaTeX で文書を作成するにあたって,
ピンポイントな用途で役に立つようなものについて,インターネット上で調べるよりも簡潔でわかりやすくまとめることを理念として作られたものです.
もちろん基本的な使い方について記述しても構いません.これを利用して各自の文書作成に役立ててください.

この文書はGithubで管理してあるので,各自好きなように編集することができます.
\LaTeX についての基本的な使い方や便利なコマンドなど,普段 \LaTeX で文書を書くときに役に立つことを見つけたり気づいたりしたときに,
gitやGithubを用いてmanual.texを編集してください.
そのとき,必ず自分でmanual.texをコンパイルし,出力されたpdfファイルに問題がないことを確認した上で,
gitまたはGithubでmanual.texとmanual.pdfを更新してください.

\LaTeX の環境構築については各自で行うことにしてください.
難しいようであればOverleafやCloudLaTeXなどのクラウド上で利用できる \LaTeX サービスをおすすめします.

\newpage

\section{数式}

\subsection{式番号}

\verb|equation|環境や\verb|align|環境で数式を書くと自動で式番号が付与されます.
式番号をつけたくない場合は
\begin{center}
    \verb|\begin{equation*}|,\verb|\begin{align*}|
\end{center}
のように環境名の後ろにアスタリスクを追加します.
また,通常の環境においても式の各行の最後に
\begin{center}
    \verb|\notag|
\end{center}
を書くことでその行だけ番号をつけないようにすることもできます.
デフォルトでは,数式番号は

式番号を文中で参照するときは,数式環境内で参照したい式の後ろに
\begin{center}
    \verb|\label{}|
\end{center}
で好きなラベル名をつけ,文中で
\begin{center}
    \verb|\ref{}|
\end{center}
と書いて,カッコ内に参照したいラベル名を入力すれば式番号を参照することができます.
しかし,この方法では参照した式番号にカッコがつかないので手動で\verb|(\ref{})|のようにしなければなりません.
そこで,プリアンプルに
\begin{center}
    \verb|\usepackage{amsmath}|
\end{center}
と書いておき,\verb|\ref{}|の代わりに
\begin{center}
    \verb|\eqref{}|
\end{center}
を使えば自動で式番号にカッコをつけることができるため,こちらを使用することをおすすめします.

また,関根研究室においては数学系の文書を作成する機会が多くなることかと思いますが,
そのときに数多く出てくる数式すべてに式番号をつけると番号が極端に大きくなる可能性があります.
しかし,その都度式番号をつけたり消したりするのは面倒なので,
自動で参照する式のみに番号を割り振るようにすると楽です.そのためには\verb|mathtools|パッケージを使います.
プリアンプルに
\begin{center}
    \verb|\usepackage{mathtools}|
\end{center}
と書いておき,さらにプリアンプルに
\begin{center}
    \verb|\mathtoolsset{showonlyrefs=true}|
\end{center}
とすれば,参照していない式には番号が振られないようになります\cite{FomulaNumber}.

\subsection{諸記号}

\subsubsection{絶対値,ノルム}

数式内で絶対値を表示したい場合,
\begin{center}
    \verb+|x|+
\end{center}
のように縦棒で絶対値にしたい箇所を挟めばよいです.
ノルムを表示したい場合,
\begin{center}
    \verb+\|x\|+
\end{center}
のようにバックスラッシュと縦棒を並べたもので挟めば$\|x\|$とすることができます.
これらのコマンドはシンプルである分融通がききやすいですが,分数を挟む場合には分数と記号の縦幅が一致しません.
これを解消するには\verb|\left|と\verb|\right|で記号を挟む必要がありますがこれをいちいちタイプするのは手間です.

\verb|physics|パッケージを使用すればこの問題を解決できます.プリアンプルに
\begin{center}
    \verb|\usepackage{physics}|
\end{center}
と書くことでphysicsパッケージを使用できます.\verb|physics|パッケージでは絶対値を表示したい場合,
\begin{center}
    \verb|\abs{x}|
\end{center}
とすれば$\abs{x}$が出力できます.また,ノルムを表示したい場合は,
\begin{center}
    \verb|\norm{x}|
\end{center}
とすれば$\norm{x}$と出力できます.
さらに数式内の文字や数字と同様に\verb|{}|の後ろに上付き文字や下付き文字をつけることができます.

%参考文献のセクションにwebサイトの引用のことを書く(bibのやつ引用)

\newpage

\section{参考文献について}

\LaTeX で文書を作成するとき,参考文献を管理する環境として\verb|thebibliograph|環境があります.
\verb|thebibliograph|環境の使い方については省略しますが,これを使うよりもBIBTEXを使用して,
文献データベースとなる.bibファイルで参考文献の管理をする方が便利でしょう.

\newpage

\addcontentsline{toc}{section}{\bibname}
\bibliographystyle{jplain}
\bibliography{manual}

\end{document}