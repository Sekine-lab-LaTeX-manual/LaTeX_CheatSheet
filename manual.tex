\documentclass[a4paper,11pt,titlepage]{jsarticle}
\usepackage[dvipdfmx]{graphicx}
\usepackage[top=25mm, bottom=25mm, left=25mm, right=25mm]{geometry}
\usepackage{listings, jlisting}
\usepackage{amsfonts,amsmath,amsthm,mathtools,amssymb}
\usepackage{comment}
\usepackage{physics}

\newcommand{\midrad}[1]{\left\langle #1 \right\rangle}

\theoremstyle{definition}
\newtheorem{dfn}{定義}[section]
\newtheorem{thm}{定理}[section]

\renewcommand{\proofname}{\textbf{証明}}

\renewcommand{\lstlistingname}{ソースコード}
\lstset{%
    language={C},
    basicstyle={\small},%
    identifierstyle={\small},%
    commentstyle={\small\itshape},%
    keywordstyle={\small\bfseries},%
    ndkeywordstyle={\small},%
    stringstyle={\small\ttfamily},
    frame={tb},
    breaklines=true,
    columns=[l]{fullflexible},%
    xrightmargin=0zw,%
    xleftmargin=3zw,%
    numberstyle={\scriptsize},%
    numbersep=1zw,%
    lineskip=-0.5ex%
}

\begin{document}

\title{\LaTeX マニュアル}
\author{関根研究室}
\date{\today}
\maketitle

\tableofcontents

\newpage

\section{この文書について}

このマニュアルは \LaTeX で文書を作成するにあたって,
ピンポイントな用途で役に立つようなものについて,インターネット上で調べるよりも簡潔でわかりやすくまとめることを理念として作られたものです.
もちろん基本的な使い方について記述しても構いません.これを利用して各自の文書作成に役立ててください.

この文書はGithubで管理してあるので,各自好きなように編集することができます.
\LaTeX についての基本的な使い方や便利なコマンドなど,普段 \LaTeX で文書を書くときに役に立つことを見つけたり気づいたりしたときに,
gitやGithubを用いてmanual.texを編集してください.
そのとき,必ず自分でmanual.texをコンパイルし,出力されたpdfファイルに問題がないことを確認した上で,
gitまたはGithubでmanual.texとmanual.pdfを更新してください.

\LaTeX の環境構築については各自で行うことにしてください.
難しいようであればOverleafやCloudLaTeXなどのクラウド上で利用できる \LaTeX サービスをおすすめします.

\section{数式}

\subsection{諸記号}

\subsubsection{絶対値,ノルム}

数式内で絶対値を表示したい場合,
\begin{center}
    \verb+|x|+
\end{center}
のように縦棒で絶対値にしたい箇所を挟めばよいです.
ノルムを表示したい場合,
\begin{center}
    \verb+\|x\|+
\end{center}
のようにバックスラッシュと縦棒を並べたもので挟めば$\|x\|$とすることができます.
これらのコマンドはシンプルである分融通がききやすいですが,分数を挟む場合には分数と記号の縦幅が一致しません.
これを解消するには\verb|\left|と\verb|\right|で記号を挟む必要がありますがこれをいちいちタイプするのは手間です.

physicsパッケージを使用すればこの問題を解決できます.プリアンプルに
\begin{center}
    \verb|\usepackage{physics}|
\end{center}
と書くことでphysicsパッケージを使用できます.physicsパッケージでは絶対値を表示したい場合,
\begin{center}
    \verb|\abs{x}|
\end{center}
とすれば$\abs{x}$が出力できます.また,ノルムを表示したい場合は,
\begin{center}
    \verb|\norm{x}|
\end{center}
とすれば$\norm{x}$と出力できます.
さらに数式内の文字や数字と同様に\verb|{}|の後ろに上付き文字や下付き文字をつけることができます.
\end{document}